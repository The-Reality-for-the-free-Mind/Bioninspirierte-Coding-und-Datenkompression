\documentclass{uni}
\usepackage{enumerate}

\begin{document}


% Parameters for \maketitle 
% #1: Number of the current home assignment
% #2: Date the assignment is due
% #3: Name of tutor
% #4: Number of the tutorial
% #5: Name of the first group member
% #6: Name of the second group member
% #7: Name of the third  group member



% \maketitle{Nummer}{Abgabedatum}{Tutor-Name}{Gruppennummer}
%           {Teilnehmer 1}{Teilnehmer 2}{Teilnehmer 3}
\maketitle{1}{06.11.2012}{Konrad Gadzicki und Tobias Kluth}{}
          {Liuqing Yang}{}{}

\section{Entropie und Redundanz}
Eine Quelle gibt Zeichen aus dem Alphabet $A = \{a_1, a_2, a_3, a_4, a_5\}$ mit den Wahrscheinlichkeiten $P(a_1) = 0.15, P(a_2) = 0.04, P(a_3) = 0.26, P(a_4) = 0.05$ und $P(a_5) = 0.5$

\begin{enumerate}[a)]
	\item Berechnet die Entropie der Quelle
	\item Bestimmt einen Huffman Code f\"ur diese Quelle
	\item Bestimmt die durchschnittliche Codewortl�nge f�r den Code in b) und berechnet die Redundanz.
\end{enumerate}

\section{Unfaire M�nzen und Informationsgehalt}
Betrachtet hierzu ein einfaches M�nzwurf-Experiment, welches bekannterma�en nur
zwei unterschiedliche Resultate haben kann: Kopf oder Zahl. Berechnet zun�chst die Entropie
f�r den Wurf einer "`fairen"' M�nze. Simuliert danach eine gezinkte M�nze, indem Ihr die Wahrscheinlichkeit $p$ f�r eines der Elementarereignisse in Schritten von 0.05 von 0.05 bis 0.95 variiert
(Warum nicht zwischen 0 und 1?) und dem anderen Ereignis jeweils eine Wahrscheinlichkeit
von 1-p zuweist. Bestimmt die Entropie eines M�nzwurfexperiments f�r diese unterschiedlichen
Wahrscheinlichkeiten und plottet die Resultate. (Tragt auf der x-Achse die Wahrscheinlichkeit
f�r eines der Elementarereignisse und auf der y-Achse die dazugeh�rige Entropie auf.)\\
\\
Was f�llt auf? Bei welcher Wahrscheinlichkeit ist die Entropie des Experiments am h�chsten?
Wie ist dies zu erkl�ren? Bitte gebt Eure L�sung inklusive dem Plot der Resultate und dem
dazugeh�rigen Code ab.
\section{Huffman Codes}
Implementiert ein Programm, welches f�r ein gegebenes Alphabet und den dazugeh�rigen Wahrscheinlichkeiten einen Huffman-Code mit minimaler Varianz der Codewortl�ngen generiert. Achtet bei Eurer Implementierung darauf, dass diese m�glichst generisch ist, da diese im Laufe des Semesters f�r unterschiedliche Symboltypen wiederverwendet werden soll.\\
\\
Der Konstruktor Eurer Klasse bekommt ein Array �bergeben, wobei jedes Array-Element ein Symbol darstellt. Aus dem Array sollen die Auftrittswahrscheinlichkeiten der Symbole berechnet und hiermit ein Huffman Code bestimmt werden.\\
\\
Der Konstruktor soll weiter die M�glichkeit besitzen, einen �bergebenen Code zu benutzen statt diesen zu generieren.\\
\\
Eure Klasse soll au�erdem die Funktionen \textbf{encode(array)} und \textbf{decode(string)} zum Kodieren bzw. Dekodieren anbieten. Bei der Kodierung soll das eingegebene Array (die zu codierenden Daten) dem Code entsprechend in eine neue Bitfolge umgewandelt werden, die Ihr hier einfach in einem String der Form "011010" ablegen k�nnt. Die Methode decode soll einen solchen Strin dekodieren, ihn also wieder dem Code entsprechend in ein Array umwandeln.\\
\\
Testet Eure Implementierung mit dem in der ersten �bung behandelten Text. Berechnet die Wahrscheinlichkeiten f�r das Auftreten einzelner Zeichen sowie die Wahrscheinlichkeiten f�r das Auftreten von m�glichen 2er-Bl�cken ebendieser Zeichen, wie dies in der �bung gezeig Generiert mit dem von Euch implementierten Programm jeweils einen Human Code und ver-gleicht die durchschnittlichen Wortl�ngen, Entropien und Redundanzen der beiden resultierende Codes und beschreibt und erkl�rt Eure Beobachtungen.\\
\\
Optional k�nnt Ihr (zus�tzlich) mit Hilfe eines regul�ren Ausdrucks alle Zeichen au�er den Buch-staben von a-h durch Leerzeichen ersetzen und die Ezienz des resultierenden Codes mit der Ezienz des Codes vergleichen, den Ihr in der �bung von Hand erstellt habt.
\end{document}


